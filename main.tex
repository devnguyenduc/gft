\documentclass{article}
\usepackage[utf8]{inputenc}
\title{}

\begin{document}

\maketitle

\section{Introduction to fluids and related concepts}
\subsection{Element, fluids}
\subsubsection{Definitions}
In general, a ‘fluid’ is a special kind of continuum. A continuum is a collection of particles so numerous that the dynamics of individual particles cannot be followed, leaving only a description of the collection in terms of ‘average’ or ‘bulk’ quantities: number of particles per unit volume, density of energy, density of momentum, pressure, temperature, etc. The behavior of a lake of water, and the gravitational field it generates, does not depend upon where any one particular water molecule happens to be: it depends only on the average properties of huge collections of molecules.

These properties can vary from point to point in the lake: the pressure is larger at the bottom than at the top, and the temperature may vary as well. The atmo-sphere, another fluid, has a density that varies with position. This raises the question of how large a collection of particles to average over: it must clearly be large enough so that the individual particles don’t matter, but it must be small enough so that it is relatively homogeneous: the average velocity, kinetic energy, and interparticle spacing must be the same everywhere in the collection. Such a collection is called an ‘element’.

\subsubsection{Properties of fluid}

So far, this notion of a continuum embraces rocks as well as gases. A fluid is a continuum that ‘flows’: this definition is not very precise, and so the division between solids and fluids is not very well defined. Most solids will flow under high enough pressure. What makes a substance rigid? After some thought we should be able to see that rigidity comes from forces parallel to the interface between two elements. Two adjacent elements can push and pull on each other, but the continuum won’t be rigid unless they can also prevent each other from sliding along their common boundary. A fluid is characterized by the weakness of such antislipping forces compared to the direct push–pull force, which is called pressure.
\subsection{Perfect fluid, dust}
\subsubsection{Perfect fluid}
A perfect fluid is defined as one in which all anti slipping forces are zero, and the only force between neighboring fluid elements is pressure
\subsubsection{Dust}
We will introduce the relativistic description of a fluid with the simplest one: ‘dust’ is defined to be a collection of particameles, all of which are at rest in some one Lorentz frame (MCRF)

$ n := $  number density in the MCRF of the element.
\\
$\frac{n}{\sqrt{1-v^2}} = $ number density in frame in which particles have velocity $v$. 
\subsection{The number-flux four-vector.}

\subsubsection{A unification of number density and flux in a four-vertor form.}

Consider the vector N defined by:
$$ \vec{N} = n \vec{U}$$
with:
\begin{eqnarray*}
 \vec{U}_{\vec{O}} ( \frac{1}{\sqrt{1-v^2}},\frac{v^x}{\sqrt{1-v^2}},\frac{v^y}{\sqrt{1-v^2}},\frac{v^z}{\sqrt{1-v^2}} )
 \end{eqnarray*}
It follows that
\begin{eqnarray*}
 \vec{N}_{\vec{O}} ( \frac{n}{\sqrt{1-v^2}},\frac{n v^x}{\sqrt{1-v^2}},\frac{n v^y}{\sqrt{1-v^2}},\frac{n v^z}{\sqrt{1-v^2}} ) 
\end{eqnarray*}
in the Lorentz non rest Frame;\\
Finally we have:
$$ \vec{N} \dot \vec{N} = -n^2 , n = (-\vec{N} \dot \vec{N})^{1/2}. $$
\subsubsection{The flux across the surface}
The definition of flux: the flux of particles across a surface is the number crossing a unit area of that surface in a unit time. This clearly depends on the inertial reference frame (‘area’) and ‘time’ are frame-dependent concepts) and on the orientation of the surface (a surface parallel to the velocity of the particles won’t be crossed by any of them)

Before going on to discuss other properties of fluids, we should mention a useful fact. An inertial frame, which up to now has been defined by its four-velocity, can be defined also by a one-form, namely that associated with its four-velocity g( U, ). This has components
$$ U_\alpha = \eta_{\alpha \beta} U^\beta$$
or, in this frame,
$$ U_0 = -1, U_i = 0 $$
For instance, the energy of a particle whose four-momentum
is $\vec{p}$ is
$$ E = \langle dt, \vec{p} \rangle = p^0. $$

$$ E = -\vec{p} \dot \vec{U}. $$

\end{document}

\section{Cái gì đó}
\section{Cái gì đó 2}
\section{Stress-energy in General Relavity}
\subsection{Energy conditions and demonstrations}
\subsubsection{The four energy conditions and their physical meaning}
$$ $$
\subsubsection{Applications to perfect fluid}
$$ $$
